\Referat

Расчетно-пояснительная записка \pageref{LastPage}\,стр., \totfig{}\,рис., \tottab{}\,табл., \total{citnum}\,ист.

НОВОСТНЫЕ ИСТОЧНИКИ, РАНЖИРОВАНИЕ, КЛАСТЕРИЗАЦИЯ, ДОВЕРИЕ

Объектом исследования является рейтинг источников по степени доверия к ним. Объект разработки~--- программное обеспечение для мониторинга новостей с последующим ранжированием источников.

Цель работы~--- разработка и реализация метода ранжирования новостных источников по степени доверия.

Задачи, решаемые в работе:
\begin{itemize}
    \item Анализ предметной области;
    \item Разработка метода ранжирования новостных источников;
    \item Реализация выбранного метода и исследование его применимости к задаче.
\end{itemize}

В первой части работы описываются существующие подходы к решению задачи, проводится анализ возможных решений. Во второй части описываются выбранные методы и внесенные в них модификации. В третьей части описываются технологии, примененные при реализации метода и тестирование. В четвертой части проведены экспериментальные исследования характеристик метода и применимости к решаемой задаче.

Предлагаемые направления развития:
\begin{itemize}
    \item Построение тематического рейтинга источников;
    \item Ранжирование экспертов;
    \item Нахождение дубликатов и первоисточников новостей.
\end{itemize}

Поставленная цель была достигнута: метод ранжирования новостных источников по степени доверия. Были рассмотрены существующие недостатки решения и предложены пути
дальнейшего развития.
