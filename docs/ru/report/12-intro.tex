\Introduction

В современном информационном пространстве сформировалось огромное количество информационных источников. К примеру, новостной агрегатор <<Яндекс.Новости>> только в России получает более 100 тысяч новостей ежедневно от почти 7000 информационных источников.

Проблема выбора новостного источника усугубляется тем, что СМИ периодически публикуют недостоверную информацию. Существуют различные проекты по проверке новостей на достоверность, такие как stopfake.org. Поскольку такая проверка осуществляется вручную, эксперты сильно ограничены во времени и могут проверять только наиболее известные информационные источники, оставляя без внимания более мелкие, но при этом всё равно читаемые новостные сайты.

Для решения данной проблемы необходим метод распространения оценки экспертов на схожие новости менее популярных источников новостей с последующим ранжированием источников по степени доверия к ним.

Целью данной работы является разработка такого метода для ранжирования информационных источников и реализация системы мониторинга новостей с последующим ранжированием источников по степени доверия.

В рамках работы необходимо решить следующие задачи:
\begin{enumerate}
    \item Проанализировать предметную область;
    \item Разработать метод ранжирование источников;
    \item Разработать программное обеспечение;
    \item Провести исследование для выбора параметров методов.
\end{enumerate}
