\section{Ранжирование источников}
Задача ранжирования источников по степени доверии к ним заключается в определении рейтинга источника по имеющейся оценке недостоверности новостей, публикуемых источником.

Рассмотрим несколько методов такого ранжирования.

\subsection{Простое ранжирование} \label{ssec:simple-ranking}
Поскольку публикация недостоверной новости должно вести к уменьшения рейтинга источника, то можно начислять штрафные очки каждый раз, когда источник публикует недостоверную новость, причём количество начисляемых очков пропорционально степени уверенности в недостоверности новости.

Таким образом, рейтинг источника $S$ за промежуток времени $t$ можно определить как
\begin{equation}
    score_S(t)=\sum_{x\in S(t)} p_x,
\end{equation}
где $p_x$~--- уверенность в недостоверности новости.

При таком определении менее достоверные источники получают выше рейтинг, чем более достоверные.

\subsection{Ссылочное ранжирование}
Для более сложного ранжирования можно использовать информацию о связи источника с другими источниками через новости. Для данного подхода недостаточно только информации о сюжетах, необходимо так же определять дубликаты и первоисточники.

Рассмотрим граф новостей (рис.~\ref{fig:news-graph}). В результате обнаружения дубликатов и объединения новостей в сюжеты, получаем множество связей между новостями, которые представляются тремя видами направленных рёбер:
\begin{enumerate}
    \item отношение <<дублирует>>;
    \item отношение <<схожий сюжет>>;
    \item отношение <<тематическое продолжение>>.
\end{enumerate}

С каждым ребром ставится в соответствие некоторое число, означающее силу связи.

\begin{figure}[h]
    \centering
    \begin{tikzpicture}[]
        \begin{scope}[every node/.style={circle,thick,draw}]
            \node (A) at (-3.5,0) {$A (0.8)$};
            \node (B) at (0,3.5) {$B (0.42)$};
            \node (C) at (1.5,1) {$C (?)$};
            \node (E) at (5.5,-2) {$E (0.6)$};
            \node (F) at (6,3) {$F (0.9)$};
        \end{scope}

        \begin{scope}[>={Stealth[black]},
                every node/.style={fill=white,circle},
            every edge/.style={draw=gray,very thick}]
            \path[->] (A) edge node {$0.2$} (C);
            \path[->] (E) edge node {$0.3$} (C);
            \path[->] (F) edge node {$0.3$} (C);
            \path[->] (A) edge (B);
            \path[->] (A) edge (E);
            \path[->] (B) edge (F);
            \path[->] (E) edge (F);
            \path[->] (F) edge[bend left=40] (E);
            \path[<-] (E) edge ++(2,0);
            \path[->] (A) edge ++(0,-2);
            \path[<-] (A) edge ++(-2.3,0);
            \path[->] (F) edge ++(2,0);
            \path[<-] (F) edge ++(0,2);
            \path[<-] (B) edge ++(-3,0);
        \end{scope}
    \end{tikzpicture}
    \caption{Граф новостей}
    \label{fig:news-graph}
\end{figure}

Поскольку каждая новость так же связана с источником, из которого получена данная новость, есть возможность перейти к схожему графу для источников. Теперь, если пометить некоторую новость как фальсифицированную, то можно, используя установленные связи, передать часть веса остальным новостям, связанных с данной, а значит и источнику, связанному с данной новостью.

В качестве алгоритма распространения веса могут быть использованы модификации PageRank или HITS алгоритмов \cite{segaran07}.

Из-за большей сложности (требуется так же определять дубликаты и первоисточники новостей) данный метод в работе не используется.
