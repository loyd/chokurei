\chapter{Экспериментальный раздел}
\section{Постановка эксперимента}
Необходимо оценить влияние на качество кластеризации меры схожести (см.~\ref{ssec:similarity}) при кластеризации новостей.

Для оценки качества кластеризации будем использовать набор данных, собранный с ресурса <<Яндекс.Новости>>\footnote{https://news.yandex.ru/export.html} в категориях <<Политика>>, <<Финансы>>, <<Экономика>> и <<В мире>>.

Так как ресурс не предоставляет текст самих новостей, а только ссылки и небольшое описание, поэтому необходимо воспользоваться алгоритмом извлечения содержимого (см.~\ref{sssec:readability}).

Поскольку данный алгоритм может урезать релевантную информацию и не всегда верно определяет основное содержимое, то из полученного набора данных были удалены все новостные статьи короче 150 символов, в результате чего был получен набор из 2005 новостей, относящихся к 263 кластерам.

Кластеры, собранные с ресурса <<Яндекс.Новости>> будем называть образцовыми и оценивать качество кластеризации относительно них. В качестве алгоритма кластеризации будем использовать ICA (см.~\ref{ssec:clustering}).

\section{Метрики качества кластеризации}
В качестве метрик качества кластеризации применяют F-меру, чистоту и энтропию \cite{andrews07}.

Для определения F-меры необходимо сначала определить понятие точности и полноты.

Точность определяется как
\begin{equation}
    P(c,o)=\frac{|c\cap o|}{|c|},
\end{equation}
\begin{conditions}
    $c$ & множество полученных объектов; \\
    $o$ & множество релевантных объектов. \\
\end{conditions}

Полнота же определяется как
\begin{equation}
    R(c,o)=\frac{|c\cap o|}{|o|}
\end{equation}

Для объединения точности и полноты в одну метрику используют $F_1$-меру:
\begin{equation}
    F_1(c,o)=\frac{2\cdot P(c,o)\cdot R(c,o)}{P(c,o)+R(c,o)}
\end{equation}

Так как имеет место множеством кластеров, необходимо объединить показатели данной метрики в результирующую \cite{andrews07}:
\begin{equation}
    F=\sum_{o\in O}\frac{|o|}{n}\max_{c\in C}F_1(c,o),
\end{equation}
\begin{conditions}
    $C$ & множество кластеров, сформированных системой; \\
    $O$ & образцовое множество кластеров; \\
    $n$ & количество кластеризуемых объектов. \\
\end{conditions}

Для оценки доли корректно кластеризованных объектов используют метрику чистоты \cite{deepa12}:
\begin{equation}
    Purity=\frac{1}{n}\sum_{o\in O}\max_{c\in C}|o\cap c|
\end{equation}

Значение чистоты близкое к 0 указывает на плохую кластеризацию, в то время как идеальная кластеризация имеет значение 1.

А для оценки распределения объектов внутри вычисленного кластера относительно образцовых используют метрику энтропии:
\begin{equation}
    Entropy=-\frac{1}{\log k}\sum_{o\in O}\frac{1}{n}\sum_{c\in C}|o\cap c|\cdot \log P(c,o),
\end{equation}
где $k$~--- количество кластеров в образцовом множестве.

Низкое значение энтропии показывает, что вычисленные системой кластеры не содержат объектов из множества разных образцовых кластеров.

\section{Результаты эксперимента}
Сравним меры схожести новостей, представленные в~\ref{ssec:similarity}.

\begin{table}[h]
    \centering
    \begin{tabular}{l | c | c | c}
        \hline
        Мера схожести & F-метрика & Чистота & Энтропия \\ \hline\hline
        Эвклидово расстояние             & 0.89 & 0.88 & 0.105  \\ \hline
        Косинусная мера                  & 0.92 & 0.93 & 0.081  \\ \hline
        Мера Жаккара                     & 0.91 & 0.92 & 0.082  \\ \hline
        Эвклидово расстояние со штрафами & 0.89 & 0.89 & 0.104  \\ \hline
        Косинусная мера со штрафами      & 0.93 & 0.94 & 0.080  \\ \hline
        Мера Жаккара со штрафами         & 0.92 & 0.92 & 0.082  \\
        \hline
    \end{tabular}
    \caption{Сравнение качества кластеризации для разных мер схожести.}
    \label{tbl:sim-quality}
\end{table}

Исходя из результатов, представленных в таблице~\ref{tbl:sim-quality} можно сделать вывод, что лучшей мерой схожести новостей среди представленных является косинусная мера со штрафами.

\section{Выводы}
На основе результата эксперимента по влиянию мер схожести на качество кластеризации новостей, проведённого на выборе из 2005 новостей и 263 кластеров, можно сделать вывод о том, что наилучшей мерой схожести является косинусная мера со штрафами.
